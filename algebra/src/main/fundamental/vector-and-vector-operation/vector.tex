\section{vector ($\vec{v}$)}
\subsection{မှတ်စု}
{\begin{itemize}
        \item vector တွေမှာ direction နှင့် magnitude နှစ်ခုလုံးရှိကြတယ်။
        \item v အပေါ်မှာ arrow လေးထည့်ပြီးဖော်ပြလေ့ရှိကြတယ်။
        \item vector ရဲ့ entry အားလုံးဟာ real numbers ထဲကသာဖြစ်ရမယ်။
        \item vector ရဲ့ entry အရေအတွက်ဟာ vector ရဲ့ dimension အရေအတွက်ဖြစ်တယ်။
        \item vector ရဲ့ မြှားပါတဲ့ဘက်ခြမ်းသည် head ဖြစ်ပြီး ဆန့်ကျင်ဘက်က tail ဖြစ်တယ်။
        \item (0,0) က standard position, origin ဖြစ်တယ်။
    \end{itemize}}
\begin{minipage}{0.45\textwidth}
    \centering
    \import{fundamental/vector-and-vector-operation/diagram}{d1.tex}
\end{minipage}
\hfill
\begin{minipage}{0.45\textwidth}
    \centering
    \import{fundamental/vector-and-vector-operation/diagram}{d2.tex}
\end{minipage}
\clearpage
\begin{center}
    \import{fundamental/vector-and-vector-operation/diagram}{d3.tex}
\end{center}
\import{fundamental/vector-and-vector-operation}{types-of-vector.tex}
\subsection{position vector သို့ပြောင်းခြင်း}
\import{fundamental/vector-and-vector-operation/diagram}{d6.tex}
\begin{equation}
    {\color{purple} \begin{split}
        position\hspace{1mm}vector & = \vec{w}_{head} - \vec{w}_{tail} \\
                        & = (7,3) - (4,1) \\
                        & = (3,2) \\
                        & = \vec{v}
    \end{split}}
\end{equation}
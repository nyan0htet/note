\section{Vector Length $\|v\|$}
\subsection{မှတ်စု}
\begin{itemize}
    \item vector ရဲ့ tail က standard basic point ဖြစ်နေမှသာတွက်လို့ရမယ်။
    \item vector က origin ကမစလျှင် eq\ref{eq:standard-position} အတိုင်းပြောင်းပါ။
    \item $(v_1,v_2)$ ရဲ့ တစ်စိတ်တစ်ပိုင်း component $v_1$ ကို $|v_1|$ ပုံစံဖော်ပြနိုင်တယ်။
\end{itemize}
\begin{minipage}{0.45\textwidth}
    \centering
    \import{fundamental/vector-and-vector-operation/diagram}{d15.tex}
\end{minipage}
\hfill
\begin{minipage}{0.45\textwidth}
    \centering
    \import{fundamental/vector-and-vector-operation/diagram}{d16.tex}
\end{minipage}

In Fig \ref{fig:vector-and-vector-operation-d15},$\vec{v}=(v_1,v_2)\in\mathbb{R}^2$ ကို $\vec{v}=(v_1,0)+(0,v_2)$ ကနေတွက်ထုတ်ထားတာဖြစ်တယ်။
\[
    \begin{split}
        \|\vec{v}\| & =\sqrt{\|(v_1,0)\|^2+\|(0,v_2)\|^2} \\
                    & =\sqrt{|v_1|^2+\|v_2|^2}            \\
                    & =\sqrt{v_1^2+v_2^2}                 \\
                    & =\sqrt{(v_1,v_2)\cdot(v_1,v_2)}     \\
                    & =\sqrt{\vec{v}\cdot\vec{v}}
    \end{split}
\]

In Fig \ref{fig:vector-and-vector-operation-d16},$\vec{v}=(v_1,v_2,v_3)\in\mathbb{R}^3$ ကို $\vec{v}=(v_1,v_2,0)+(0,0,v_3)$ ကနေတွက်ထုတ်ထားတာဖြစ်တယ်။
\[
    \begin{split}
        \|\vec{v}\| & =\sqrt{\|(v_1,v_2,0)\|^2+\|(0,0,v_3)\|^2} \\
                    & =\sqrt{(\sqrt{v_1^2+v_2^2})^2+\|v_3\|^2}  \\
                    & =\sqrt{v_1^2+v_2^2+v_3^2}                 \\
                    & =\sqrt{(v_1,v_2,v_3)\cdot(v_1,v_2,v_3)}   \\
                    & =\sqrt{\vec{v}\cdot\vec{v}}
    \end{split}
\]

\subsection{Definition}
$\vec{v}=(v_1,v_2,\dots,v_n)\in\mathbb{R}^n$ ၏ length ကိုရှာချင်ရင်
\begin{equation}
    {\color{purple} \|\vec{v}\|\stackrel{def}{=}\sqrt{\vec{v}\cdots\vec{v}}=\sqrt{v_1^2+v_2^2+\dots+v_n^2}}
\end{equation}

\subsection{Example}
(2,-5,4,6) ၏ length ကိုရှာလျှင်

\[
    \begin{split}
        \|(2,-5,4,6)\| & = \sqrt{2^2+(-5)^2+4^2+6^2} \\
                       & = \sqrt{81}                 \\
                       & = 9
    \end{split}
\]

$(cos(\theta),sin(\theta))$ ၏ length ကိုရှာလျှင်

\[
    \begin{split}
        \|(cos(\theta),sin(\theta))\| & =\sqrt{{cos}^2(\theta)+{sin}^2(\theta)} \\
                                      & =\sqrt{1}                               \\
                                      & =1
    \end{split}
\]
\begin{center}
    \import{fundamental/vector-and-vector-operation/diagram}{d17.tex}
\end{center}
\[
    \begin{split}
        \|(1,1,1)\|=\sqrt{1^2+1^2+1^2}=\sqrt{3}
    \end{split}
\]

\subsection{Theorem}

$\vec{v}\in\mathbb{R}^n$, $c\in\mathbb{R}$ ဖြစ်လျှင်
\begin{equation}
    {\color{purple} \begin{split}
            (a)\hspace{5mm}\|c\vec{v}\|  & =|c|\cdot\|\vec{v}\|                                            \\
            (b)\hspace{6.5mm}\|\vec{v}\| & > 0,\text{\color{black} with equality if and only if }\vec{v}=0
        \end{split}}
\end{equation}

\subsection{Unit Vector ($\hat{u}=1$)}

unit vector $\hat{u}$ ရဲ့ length ကတော့ 1ရှိပါတယ်။ တစ်ခြား vector တွေကနေ unit vector ဖြစ်အောင် scaling လုပ်တာကို normalization လုပ်ခြင်းလို့ခေါ်ပါတယ်။
direction က မူရင်း vector ၏ direction အတိုင်းသာဖြစ်တယ်။ $\hat{u}\in\mathbb{R}^2$ ကတော့ unit circle ဖြစ်ပြီး $\hat{u}\in\mathbb{R}^3$ ကတော့ unit sphere ဖြစ််တယ်။
$\vec{v}\in\mathbb{R}^n$ဖြစ်လျှင်
\begin{equation}
    {\color{purple} \begin{split}
            (a)\hspace{5mm}\hat{u}     & =\frac{\vec{v}}{\|\vec{v}\|} \\
            (b)\hspace{2mm}\|\hat{u}\| & =1
        \end{split}}
\end{equation}
\begin{minipage}{0.45\textwidth}
    \centering
    \import{fundamental/vector-and-vector-operation/diagram}{d18.tex}
\end{minipage}
\hfill
\begin{minipage}{0.45\textwidth}
    \centering
    \import{fundamental/vector-and-vector-operation/diagram}{d19.tex}
\end{minipage}

$\vec{v}=(3,4)$ ကို unit vector သို့ scaling လုပ်လျှင်,
\[
    \begin{split}
        \hat{u}\quad of\quad\vec{v}      & =\frac{\vec{v}}{\|\vec{v}\|}             \\
                                         & =\frac{(3,4)}{\sqrt{3^2+4^2}}            \\
                                         & =\frac{(3,4)}{5}                         \\
                                         & =(\frac{3}{5},\frac{4}{5})               \\
        \|\hat{u}\|\quad of \quad\vec{v} & = \sqrt{(\frac{3}{5})^2+(\frac{4}{5})^2} \\
                                         & = \sqrt{\frac{9}{25}+\frac{16}{25}}      \\
                                         & = \sqrt{\frac{25}{25}}                   \\
                                         & = \sqrt{1}                               \\
                                         & = 1
    \end{split}
\]
\begin{center}

    \import{fundamental/vector-and-vector-operation/diagram}{d20.tex}
\end{center}

\subsection{Cauchy-Schwarz Inequality}
$\vec{v}\in\mathbb{R}^n$ နှင့် $\vec{w}\in\mathbb{R}^n$ တို့သည် linearly dependent ဖြစ်နေမှသာလျှင် Cauchy Inequality ကသုံးလို့ရတယ်။ $\vec{v}$ ရဲ့ head က $\vec{w}$ ၏ tail မှာဆက်နေလျင်, or, $\vec{w}$ ရဲ့ head က $\vec{v}$ ၏ tail မှာဆက်နေမှသာ Cauchy Inequality ကမှန်ကန်တယ်။
\begin{equation}
    {\color{purple} \begin{split}
            |\vec{v}\cdot\vec{w}| & \le\|\vec{v}\|\|\vec{w}\|
        \end{split}}
\end{equation}
$\vec{v}=(1,2), \vec{w}=(3,4),$
\[
    \begin{split}
        \vec{v}\cdot\vec{w}   & =(1,2)\cdot(3,4) \\
                              & =3+8             \\
                              & =11              \\
        |\vec{v}\cdot\vec{w}| & =|11|            \\
                              & =11              \\
        \|\vec{v}\|           & =\sqrt{1^2+2^2}  \\
                              & =\sqrt{5}          \\
        \|\vec{w}\|           & =\sqrt{3^2+4^2}  \\
                              & =\sqrt{25}       \\
                              & =5 \\
        \|\vec{v}\|\|\vec{w}\|&=5\sqrt{5} \\
        & \approx 11.028 \\
        |\vec{v}\cdot\vec{w}|&\le\|\vec{v}\|\|\vec{w}\|
    \end{split}
\]

\subsection{Triangle Inequality}
တြိဂံတစ်ခု၏ အနားတစ်ခုသည် ကျန်အနားနှစ်ခု ပေါင်းခြင်းထက်ငယ် သို့မဟုတ် ပေါင်းခြင်းနှင့်ညီနိုင်သည်။
\begin{equation}
    \begin{split}
        \vec{x}&=\vec{v}\cdot\vec{w} \\
        \|\vec{x}\|&\le\|\vec{v}\|+\|\vec{w}\| \\
        {\color{purple} \|\vec{v}\cdot\vec{w}\|}&{\color{purple} \le\|\vec{v}\|+\|\vec{w}\|}
    \end{split}
\end{equation}

\begin{center}
    \import{fundamental/vector-and-vector-operation/diagram}{d21.tex}
\end{center}

\subsection{The Angle between Vector}
$\vec{v},\vec{w}\in\mathbb{R}^n$ နှစ်ခုကြားက angle $\theta$ ကိုရှာချင်လျှင်, Figure\ref{fig:vector-and-vector-operation-d5}အရ $\vec{v}$ နှင့် $\vec{w}$ ကိုဆက်ထားတဲ့ vector လိုအပ်တဲ့အတွက် vector substraction လုပ်မှသာ ရနိုင်တယ်။ ချိတ်ဆက်ထားတဲ့ vector သည် $\vec{v}-\vec{w}$ ဖြစ်လာသည်။

\begin{minipage}{0.45\textwidth}
    \centering
    \import{fundamental/vector-and-vector-operation/diagram}{d22.tex}
\end{minipage}
\hfill
\begin{minipage}{0.45\textwidth}
    \centering
    \import{fundamental/vector-and-vector-operation/diagram}{d23.tex}
\end{minipage}

law of cosines အရ $\theta,\vec{v},\vec{w},\vec{v}-\vec{w}$ တို့ကို
\subsection{Definition}
\begin{equation}
    \begin{split}
        \|\vec{v}-\vec{w}\|^2&=\|\vec{v}\|^2+\|\vec{w}\|^2-2\|\vec{v}\|\|\vec{w}\|\cos(\theta) \\
        \|\vec{v}-\vec{w}\|^2&=(\sqrt{(\vec{v}-\vec{w})^2})^2 \\
        &=(\vec{v}-\vec{w})^2 \\
        &=(\vec{v}-\vec{w})\cdot(\vec{v}-\vec{w}) \\
        &=\vec{v}\cdot\vec{v}-\vec{v}\cdot\vec{w}-\vec{v}\cdot\vec{w}+\vec{w}\cdot\vec{w} \\
        &=\vec{v}^2-2(\vec{v}\cdot\vec{w})+\vec{w}^2 \\
        &=\|\vec{v}\|^2-2(\vec{v}\cdot\vec{w})+\|\vec{w}\|^2 \\
        \|\vec{v}\|^2-2(\vec{v}\cdot\vec{w})+\|\vec{w}\|^2&=\|\vec{v}\|^2+\|\vec{w}\|^2-2\|\vec{v}\|\|\vec{w}\|\cos(\theta)  \\
        \vec{v}\cdot\vec{w}&=\|\vec{v}\|\|\vec{w}\|\cos(\theta) \\
        {\color{purple} cos(\theta)}&={\color{purple} \frac{\vec{v}\cdot\vec{w}}{\|\vec{v}\|\|\vec{w}\|}} \\
        {\color{purple} \theta}&={\color{purple} cos^{-1}(\frac{\vec{v}\cdot\vec{w}}{\|\vec{v}\|\|\vec{w}\|})}
    \end{split}
\end{equation}
\subsection{Example}
$\vec{v}=(1,2),\vec{w}=(3,4)$ တို့ကြားက angle $\theta$ ကိုရှာလျှင်
\[
    \begin{split}
        \theta&=cos^{-1}(\frac{(1,2)\cdot(3,4)}{\|(1,2)\|\|(3,4)\|}) \\
        &=cos^{-1}(\frac{9}{5\sqrt{5}}) \\
        &\approx 0.1799 radian \\
        &\approx 10.30^{\circ}
    \end{split}
\]
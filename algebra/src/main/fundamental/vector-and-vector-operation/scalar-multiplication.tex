\section{scalar multiplication (c$\vec{v}$)}
\subsection{မှတ်စု}
\begin{itemize}
    \item $|c|>1$ ဖြစ်လျှင် $\vec{v}$ သည် stretch ဖြစ်မည်။
    \item $|c|<1$ ဖြစ်လျှင် $\vec{v}$ သည် shrink ဖြစ်မည်။
    \item $c<0$ ဖြစ်လျှင် $\vec{v}$ ရဲ့ direction ကပြောင်းပြန်ဖြစ်သွားမည်။
\end{itemize}
\subsection{Definition}
\begin{equation}
    {\color{purple} c\vec{v}\stackrel{def}{=}(cv_1,cv_2,\dots,cv_n)}
\end{equation}
\begin{center}
    \import{fundamental/vector-and-vector-operation/diagram}{d9.tex}
\end{center}

\subsection{Theorem}
$\vec{v}, \vec{w} \in \mathbb{R}^n$ များသည် vectors, $c, d \in \mathbb{R}$ များသည် scalars များဖြစ်သည်။
\begin{equation}
    {\color{purple}
        \begin{split}
            (a) \hspace{2mm} c(\vec{v}+\vec{w}) & =c\vec{v}+c\vec{w} \\
            (b) \hspace{3mm} (c+d)\vec{v}       & =c\vec{v}+d\vec{v} \\
            (c) \hspace{7mm} c(d\vec{v})        & =(cd)\vec{v}
        \end{split}}
    \label{eq:scalar-multiplication-theorem}
\end{equation}
%example
\subsection{Example}
In \ref{fig:vector-and-vector-operation-d10}, $\vec{v}=(2,1,-1)$, $\vec{w}=(-1,0,3)$, $3\vec{v}-2\vec{w}=?$
\begin{center}
    \import{fundamental/vector-and-vector-operation/diagram}{d10.tex}
\end{center}
\[
    \begin{split}
        3\vec{v}-2\vec{w} & =3(2,1,-1)-2(-1,0,3) \\
                          & = (6,3,-3)-(-2,0,6)  \\
                          & = (6+2,3-0,-3-6)     \\
                          & = (8,3,-9)
    \end{split}
\]
ပုံ\ref{fig:vector-and-vector-operation-d11} မှာ,hexagon ရဲ့ (0,0) ကနေ သူရဲ့ထောင့်တွေဆီကို သွားတဲ့ vecotr 6 ခု ကိုပေါင်းရင်, vector တစ်ခုကတော့ (1,0)
\begin{center}
    \import{fundamental/vector-and-vector-operation/diagram}{d11.tex}
\end{center}
\[
    \begin{split}
        \vec{v}+\vec{w}+\vec{x}+(-\vec{v})+(-\vec{w})+(-\vec{x})=0
    \end{split}
\]
equation တွေကနေ$\vec{x}$ ကိုရှာခြင်း,
\[
    \begin{split}
        (a)\hspace{1cm}\vec{x}-(3,2,1) & =(1,2,3)-3\vec{x}         \\
        \vec{x}                        & =(3,2,1)+(1,2,3)-3\vec{x} \\
        \vec{x}+3\vec{x}               & =(3+1,2+2,1+3)            \\
        4\vec{x}                       & =(4,4,4)                  \\
        \vec{x}                        & =\frac{1}{4}(4,4,4)       \\
        \vec{x}                        & =(1,1,1)
    \end{split}
\]
\[
    \begin{split}
        (b)\hspace{1cm}\vec{x}+2(\vec{v}+\vec{w}) & =-\vec{v}-3(\vec{x}-\vec{w})   \\
        x+2\vec{v}+2\vec{w}                       & =-\vec{v}-3\vec{x}+3\vec{w}    \\
        4\vec{x}                                  & =-3\vec{v}+\vec{w}             \\
        \vec{x}                                   & =\frac{1}{4}(3\vec{v}+\vec{w})
    \end{split}
\]
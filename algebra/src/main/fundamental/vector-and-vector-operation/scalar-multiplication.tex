\section{scalar multiplication (c$\vec{v}$)}
\subsection{မှတ်စု}
\begin{itemize}
    \item $|c|>1$ ဖြစ်လျှင် $\vec{v}$ သည် stretch ဖြစ်မည်။
    \item $|c|<1$ ဖြစ်လျှင် $\vec{v}$ သည် shrink ဖြစ်မည်။
    \item $c<0$ ဖြစ်လျှင် $\vec{v}$ ရဲ့ direction ကပြောင်းပြန်ဖြစ်သွားမည်။
\end{itemize}
\subsection{Definition}
\begin{equation}
    {\color{purple} c\vec{v}\stackrel{def}{=}(cv_1,cv_2,\dots,cv_n)}
\end{equation}
\begin{center}
    \import{fundamental/vector-and-vector-operation/diagram}{d9.tex}
\end{center}
\subsection{Theorem}
$\vec{v}, \vec{w} \in \mathbb{R}^n$ များသည် vectors, $c, d \in \mathbb{R}$ များသည် scalars များဖြစ်သည်။
\begin{equation}
    {\color{purple}
        \begin{split}
            (a) \hspace{2mm} c(\vec{v}+\vec{w})& =c\vec{v}+c\vec{w} \\
            (b) \hspace{3mm} (c+d)\vec{v}& =c\vec{v}+d\vec{v} \\
            (c) \hspace{7mm} c(d\vec{v})& =(cd)\vec{v}
        \end{split}}
        \label{eq:scalar-multiplication-theorem}
\end{equation}
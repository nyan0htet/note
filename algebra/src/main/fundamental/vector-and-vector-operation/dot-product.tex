\section{Dot Product}
\subsection{မှတ်စု}
\begin{itemize}
    \item dimension တူတဲ့ vector အချင်းချင်းသာ dot product ရှာလို့ရတယ်။
    \item dot product ရဲ့ရလဒ်က vector မဟုတ်ပဲ number ဖြစ်တယ်။
    \item vector နှစ်ခုရဲ့ dot product သာရှိနိုင်တယ်။
    \item $\vec{v}\cdot(\vec{w}\cdot\vec{x})$ တွင် $\vec{w}\cdot\vec{x}$ သည် number ဖြစ်တဲ့အတွက် $\vec{v}\cdot$number ကို dot produt လုပ်မရပါ။
\end{itemize}
\subsection{Definition}
\begin{equation}
    {\color{purple} \vec{v}\cdot\vec{w}\stackrel{def}{=}\vec{v}_1\vec{w}_1+\vec{v}_2\vec{w}_2+\dots+\vec{v}_n\vec{w}_n}
    \label{eq:dot-product}
\end{equation}
\begin{center}
    \import{fundamental/vector-and-vector-operation/diagram}{d14.tex}
\end{center}
\subsection{Theorem}
\begin{equation}
    {\color{purple} \begin{split}
            (a)\hspace{15mm}\vec{v}\cdot\vec{w}          & =\vec{w}\cdot\vec{v}                     \\
            (b)\hspace{6mm}\vec{v}\cdot(\vec{w}+\vec{x}) & =\vec{v}\cdot\vec{w}+\vec{v}\cdot\vec{x} \\
            (c)\hspace{11mm}\vec{v}\cdot(c\vec{w})       & =c(\vec{v}\cdot\vec{w})
        \end{split}}
    \label{eq:dot-product-theorem}
\end{equation}
\subsection{Example}
$(1,2,3)$နှင့်$(4,-3,2)$ တို့၏ dot product ကိုရှာလျှင်
\[
    \begin{split}
        (1,2,3)\cdot(4,-3,2) & =1\cdot4+2\cdot(-3)+3\cdot2 \\
                             & =4-6+6                      \\
                             & =4
    \end{split}
\]
$(1,2)\in\mathbb{R}^2$နှင့်$(2,5,3)\in\mathbb{R}^3$တို့သည် dimension မတူသောကြောင့် dot product ရှာလို့မရပါ။
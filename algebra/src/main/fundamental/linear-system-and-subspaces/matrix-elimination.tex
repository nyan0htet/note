\section{Matrix Elimination}
\subsection{Gaussian elimination}
matrix ကို row echelon form ပြောင်းပြီး back substitution လုပ်၍ x, y, z တို့ရှာပြီး linear system များ၏ solution ကိုရှာခြင်းဖြစ်သည်။
$x+y=2, 2x-y=1, -x+2y=3$ ၏ solution(intersection) ကိုရှာလျှင်, 
\[
        \left[\begin{array}{c c | c}
            1 & 1 & 2 \\
            2 & -1 & 1 \\
            -1 & 2 & 3
        \end{array}\right]\xrightarrow[Row_3+=Row_1]{Row_2-=2Row_1}\left[\begin{array}{c c | c}
            1 & 1 & 2 \\
            {\color{red}0} & -3 & -3 \\
            {\color{red}0} & 3 & 5
        \end{array}\right]\xrightarrow{Row_3-=Row_2}\stackrel{Row EchelonForm}{\left[\begin{array}{c c | c}
            1 & 1 & 2 \\
            {\color{red}0} & -3 & -3 \\
            {\color{red}0} & {\color{red}0} & 2
        \end{array}\right]}
\]
$-3y=-3,x+y=2$ ပုံစံဖြစ်လာပြီး back substitution လုပ်လျှင် $0+0=2$ သည်တော့မဖြစ်နိုင်ပါ။ ထို့ကြောင့် solution (intersection) မရှိပါ။

$x+y=2, 2x-y=1, -x+2y=1$ ၏ solution(intersection) ကိုရှာလျှင်, 
\[
        \left[\begin{array}{c c | c}
            1 & 1 & 2 \\
            2 & -1 & 1 \\
            -1 & 2 & 1
        \end{array}\right]\xrightarrow[Row_3+=Row_1]{Row_2-=2Row_1}\left[\begin{array}{c c | c}
            1 & 1 & 2 \\
            {\color{red}0} & -3 & -3 \\
            {\color{red}0} & 3 & 3
        \end{array}\right]\xrightarrow{Row_3-=Row_2}\stackrel{Row EchelonForm}{\left[\begin{array}{c c | c}
            1 & 1 & 2 \\
            {\color{red}0} & -3 & -3 \\
            {\color{red}0} & {\color{red}0} & {\color{red}0}
        \end{array}\right]}
\]
back substitution လုပ်လျှင်
\[
    \begin{split}
        -3y&=-3 \\
        y&=1 \\
        x+y&=2 \\
        x+1&=2 \\
        x&=1
    \end{split}
\]
\subsection{Gauss-Jordan Elimination}
matrix ကို Reduced row echelon form ပုံစံပြောင်း၍ တတ်နိုင်သမျှ back substitution မလုပ်ပဲ solution ကိုရှာရန်ဖြစ်သည်။
$x+y=2, 2x-y=1, -x+2y=1$ ၏ solution(intersection) ကိုရှာလျှင်, 
\[
    \begin{split}
        \left[\begin{array}{c c | c}
            1 & 1 & 2 \\
            2 & -1 & 1 \\
            -1 & 2 & 1
        \end{array}\right]&\xrightarrow[Row_3+=Row_1]{Row_2-=2Row_1}\left[\begin{array}{c c | c}
            1 & 1 & 2 \\
            {\color{red}0} & -3 & -3 \\
            {\color{red}0} & 3 & 3
        \end{array}\right]\xrightarrow{Row_3-=Row_2}\stackrel{Row EchelonForm}{\left[\begin{array}{c c | c}
            1 & 1 & 2 \\
            {\color{red}0} & -3 & -3 \\
            {\color{red}0} & {\color{red}0} & {\color{red}0}
        \end{array}\right]} \\
        &\xrightarrow[normalization]{Row_2\times=-\frac{1}{3}}\left[\begin{array}{c c | c}
            1 & {\color{red}0} & 1 \\
            {\color{red}0} & {\color{green} 1} & 1 \\
            {\color{red}0} & {\color{red}0} & {\color{red}0}
        \end{array}\right] \\
        y&=1,x=1 \\
    \end{split}
\]
\subsection*{Example Diagram}
$x+y=2, 2x-y=1, -x+2y=1$ သည် $1,1$ တွင် intersection ရှိတယ်။
\begin{center}
    \import{fundamental/linear-system-and-subspaces/diagram}{d4.tex}
\end{center} 
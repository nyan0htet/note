\section{Linear Transformation $T(\vec{v})$}
\subsection{မှတ်စု}
\begin{itemize}
    \item linear transformation မှာ rotate, stretch, shrink, reflect ဖြစ်တာတွေပါဝင်တယ်။
    \item vector နှင့် matrix တွေကို linear transformation လုပ်လို့ရတယ်။
\end{itemize}
\begin{minipage}{0.45\textwidth}
    \centering
    \import{fundamental/matrix-and-matrix-operation/diagram}{d1.tex}
\end{minipage}
\hfill
\begin{minipage}{0.45\textwidth}
    \centering
    \import{fundamental/matrix-and-matrix-operation/diagram}{d2.tex}
\end{minipage}
\subsection{Definition}
$e_1,e_2,\dots,e_n$ သည် standard basic vector ဖြစ်လျှင် 
\begin{equation}
    {\color{purple} \begin{split}
        (a)\hspace{8mm} \vec{v}&=v_1e_1+v_2e_2+\dots+v_ne_n \\
        (b)\hspace{3mm} T(\vec{v})&=T(v_1e_1+v_2e_2+\dots+v_ne_n) \\
        &=v_1T(e_1)+v_2T(e_2)++v_nT(e_n)
    \end{split}}
\end{equation}
\subsection{Theorem}
$T:\m{R}^n\rightarrow\m{R}^m$ ဖြစ်တဲ့ function နှင့် linear transformation ဖြစ်လျှင်
\begin{equation}
    {\color{purple} \begin{split}
        (a)\hspace{5mm}T(\vec{v}+\vec{w})&=T(\vec{v})+T(\vec{w}) \\
        (b)\hspace{10.5mm}T(c\vec{v})&=cT(\vec{v})
    \end{split}}
    \label{eq:linear-transformation}
\end{equation}
\subsection{Example}
function တွေသည် linear transformation ဖြစ်သည်ကိုရှာလျှင်,

$T(v_1,v_2)=(1+v_1,2+v_2)$ function သည် linear transformation ဖြစ်သည်ကိုရှာလျှင် eq\ref{eq:linear-transformation} (b) အရ $v_1=0,v_2=0,c=2$ ကို အစားထိုးကြည့်လျှင်,
\[
    \begin{split}
        T(c\vec{v})&=T(c(v_1,v_2)) \\
        &=T(2(0,0)) \\
        &=T(0,0) \\
        &=(1+0,2+0) \\
        &=(12,2) \\
        cT(\vec{v})&=cT(v_1,v_2) \\
        &=2(1,2) \\
        &=(2,4) \\
        T(c\vec{v})&\ne cT(\vec{v}) \text{ (linear transformation function မဟုတ်ပါ)}
    \end{split}
\]
$T(v_1,v_2)=(v_1-v_2,v_1+v_2)$ function သည် linear transformation ဖြစ်သည်ကိုရှာလျှင် eq\ref{eq:linear-transformation} (b) အရ $v_1=1,v_2=1,c=2$ ကို အစားထိုးကြည့်လျှင်,
\[
    \begin{split}
        T(c\vec{v})&=T(c(v_1-v_2,v_1+v_2)) \\
        &=T(2(1,1)) \\
        &=T(2,2) \\
        &=(2-2,2+2) \\
        &=(0,4) \\
        cT(\vec{v})&=cT(v_1-v_2,v_1+v_2) \\
        &=2(1-1,1+1) \\
        &=2(0,2) \\
        &=(0,4) \\
        T(c\vec{v})&\ne cT(\vec{v}) \text{ (linear transformation function ဖြစ်သည်)}
    \end{split}
\]
$\m{R}^2 \rightarrow\m{R}^2$ linear transformation $T(e_1)=(1,1),T(e_2)=(-1,1)$ ဖြစ်ပြီး $T(2,3)$ ကိုရှာလျှင်
\[
    \begin{split}
        (2,3)&=2e_1+3e_2 \\
        T(\vec{v})&=T(v_1e_1+v_2e_2) \\
        &=T(2e_1+3e_2) \\
        &=2T(e_1)+3T(e_2) \\
        &=2(1,1)+3(-1,1) \\
        &=(2,2)+(-3,3) \\
        &=(-1,5)
    \end{split}
\]
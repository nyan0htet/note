\subsection{projection အရိပ်ကျခြင်း $P(\hat{u})\vec{v}$}
\begin{equation}
    {\color{purple}\begin{split}
        \hat{u}&=\frac{\vec{v}}{\|\vec{v}\|} \\
        P_u(\hat{u})&=\hat{u}{\hat{u}}^T \\
        P_u(v)&=P_u(\hat{u})\vec{v}
    \end{split}}
\end{equation}
project a vector onto a line defined by {\color{red}{\large its}} unit vector. $\vec{v}=(1,2,3)$ ဖြစ်ပြီး unit vector ၏ line ပေါ် projection ကျတာကို ရှာလျှင်
$\vec{v}$ ၏ unit vector $\hat{u}$ ကိုရှာလျှင်
\[
    \begin{split}
        \|\vec{v}\|&=\sqrt{1^2+2^2+3^2} \\
        &=\sqrt{14} \\
        \hat{u}&=\frac{1}{\sqrt{14}}(1,2,3) \\
        P_u(\hat{u})&=\hat{u}{\hat{u}}^T \\
        &=(\frac{1}{\sqrt{14}}\mthreeone{1}{2}{3})(\frac{1}{\sqrt{14}}\monethree{1}{2}{3}) \\
        &=\frac{1}{14}\mthreethree{1 \times 1}{2 \times 1}{3 \times 1}{1 \times 2}{2 \times 2}{3 \times 2}{1 \times 3}{2 \times 3}{3 \times 3} \\
        &=\frac{1}{14}\mthreethree{1}{2}{3}{2}{4}{6}{3}{6}{9} \\
        P_u(\vec{v})&=P_u(\hat{u})\vec{v} \\
        &=\frac{1}{14}\mthreethree{1}{2}{3}{2}{4}{6}{3}{6}{9}\mthreeone{1}{2}{3} \\
        &=\frac{1}{14}\mthreeone{1\times 1+2\times 2+3\times 3}{2\times 1+4\times 2+6\times 3}{3\times 1+6\times 2+9\times 3} \\
        &=\frac{1}{14}\mthreeone{14}{28}{52} \\
        &=\mthreeone{1}{2}{3} \text{(သူ့ရဲ့$\hat{u}$အပေါ်သူ projection ဖြစ်တော့ မပြောင်းလဲ)}
    \end{split}
\]
project a vector onto a line defined by unit vector of another vector. $\vec{v}=(1,2,3)$ ဖြစ်ပြီး $\vec{w}=(1,3,2)$ ၏ line ပေါ် projection ကျတာကို ရှာလျှင်
$\vec{w}$ ၏ unit vector $\hat{u}$ ကိုရှာလျှင်
\[
    \begin{split}
        \|\vec{w}\|&=\sqrt{1^2+3^2+2^2} \\
        &=\sqrt{14} \\ 
        \hat{u}_{\vec{w}}&=\frac{1}{\sqrt{14}}(1,3,2) \\
        P_u(\vec{w})\vec{v}&=\frac{1}{14}\mthreeone{1}{3}{2}\monethree{1}{3}{2}\mthreeone{1}{2}{3} \\
        &=\frac{1}{14}\mthreethree{1}{3}{2}{3}{9}{6}{2}{6}{4}\mthreeone{1}{2}{3} \\
        &=\frac{1}{14}\mthreeone{13}{39}{26} \\
        &=\mthreeone{\frac{13}{14}}{\frac{39}{14}}{\frac{26}{14}}
    \end{split}
\]
\begin{center}
    \import{fundamental/matrix-and-matrix-operation/diagram}{d5.tex}
\end{center}
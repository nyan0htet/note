\section{Matrix addition ($\m{A}+\m{B}$)}
\subsection{Definition}
$\m{A},\m{B}\in\m{M}_{m,n}$သည် matrix, $c\in\m{R}$သည် scalar, $1 \le i \le m,1 \le j \le n$ ဖြစ်လျှင်
\begin{equation}
    {\color{purple} \begin{split}
        (a)\hspace{3mm}[\m{A}+\m{B}]_{i,j}&=a_{i,j}+b_{i,j} \\
        (b)\hspace{8.25mm}[cA]_{i,j}&=ca_{i,j}
    \end{split}}
\end{equation}
$\m{A},\m{B}\in\m{M}_{m,n}$ ဖြစ်လျှင်,
\[
    \begin{split}
        \m{A}+\m{B}&=\begin{bmatrix}
            a_{1,1} & a_{1,2} & \dots & a_{1,n} \\[4pt]
            a_{2,1} & a_{2,2} & \dots & a_{2,n} \\[4pt]
            \vdots & \vdots & \ddots & \vdots \\[4pt]
            a_{m,1} & a_{m,2} & \dots & a_{m,n}
        \end{bmatrix}+\begin{bmatrix}
            b_{1,1} & b_{1,2} & \dots & b_{1,n} \\[4pt]
            b_{2,1} & b_{2,2} & \dots & b_{2,n} \\[4pt]
            \vdots & \vdots & \ddots & \vdots \\[4pt]
            b_{m,1} & b_{m,2} & \dots & b_{m,n}
        \end{bmatrix} \\
        &=\begin{bmatrix}
            a_{1,1}+b_{1,1} & a_{1,2}+b_{1,2} & \dots & a_{1,n}+b_{1,n} \\[4pt]
            a_{2,1}+b_{2,1} & a_{2,2}+b_{2,2} & \dots & a_{2,n}+b_{2,n} \\[4pt]
            \vdots & \vdots & \ddots & \vdots \\[4pt]
            a_{m,1}+b_{m,1} & a_{m,2}+b_{m,2} & \dots & a_{m,n}+b_{m,n}
        \end{bmatrix}
    \end{split}
\]
\subsection{Theorem}
$\m{A},\m{B},\m{C}\in\m{M}_{m,n}$ နှင့် $c,d\in\m{R}$ ဖြစ်လျှင်
\begin{equation}
    {\color{purple} \begin{split}
        (a)\hspace{10.5mm}\m{A}+\m{B}&=\m{B}+\m{A} \\
        (b)\hspace{1mm}(\m{A}+\m{B})+\m{C}&=\m{A}+(\m{B}+\m{C}) \\
        (c)\hspace{6.25mm}c(\m{A}+\m{B})&= c\m{A}+c\m{B} \\
        (d)\hspace{6.5mm}(c+d)\m{A}&=c\m{A}+d\m{A} \\
        (e)\hspace{11mm}c(d\m{A})&=(cd)\m{A}
    \end{split}}
\end{equation}
\subsection{Example}
$\amtwotwo{1}{2}{3}{-1}{A}{4},\amtwotwo{2}{0}{1}{1}{B}{4}$ and $\amtwothree{1}{0}{0}{-1}{1}{1}{C}{4}$ဖြစ်လျှင်, \\[1ex]
$\m{A}+\m{B}$ ကိုရှာလျှင်
\[
    \begin{split}
        \m{A}+\m{B}&=\mtwotwo{1+2}{2+0}{3+1}{-1+1} \\[1ex]
        &=\mtwotwo{3}{2}{4}{0}
    \end{split}
\]
$2\m{A}-3\m{B}$ ကိုရှာလျှင်
\[
    \begin{split}
        2\m{A}-3\m{B}&=\mtwotwo{2}{4}{6}{-2}-\mtwotwo{6}{0}{3}{3} \\
        &= \mtwotwo{2-6}{4-0}{6-3}{-2-3} \\
        &= \mtwotwo{-4}{4}{3}{-5}
    \end{split}
\]